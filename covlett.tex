\documentclass[12pt,a4paper,unicode]{moderncv}
\moderncvtheme[grey]{classic}                % oldstyle banking casual classic
\usepackage[utf8]{inputenc}
\usepackage[scale=0.80]{geometry}
\setlength{\hintscolumnwidth}{3cm}

%\hypersetup{colorlinks,breaklinks,urlcolor=linkcolour, linkcolor=linkcolour}

\firstname{Lee James}
\familyname{O'Riordan}
%\title{BSc(Hons), AMInstP}
\address{Quantum Systems Unit, Okinawa Institute of Science and Technology Graduate University}{1919-1 Tancha, Onna-son, Okinawa, Japan 904-0495}
\mobile{+81 (0)80 6498 4323}
\email{lee.oriordan@oist.jp}
\social[linkedin]{loriordan}
\social[github]{mlxd}

\makeatletter

\begin{document}
    \recipient{The people}{The location.}
    \date{\today} % Letter date
    \opening{Dear Sir or Madam,}
    \closing{Sincerely,}
    \makelettertitle
{
    I am writing to express interest in the position advertised as ``POSXYZ''. Coming from a physics background, I have strong skills in computational work that I know will be beneficial and apt for the role. My area of specialization is in theoretical cold atomic physics, with an emphasis on numerical solutions of complex differential equations and linear algebra problems.

    I have worked extensively on developing high-performance algorithms and software using CUDA. These methods have been independently performance tested, and have been shown to outperform well-established and mature software in C/C++ and \textsc{MATLAB}; I would be happy to share these results with you. These numerical techniques allowed the modeling of physical systems otherwise difficult without the use of a supercomputer, and even exceeding traditional HPC methods in specific cases. The performance of these algorithms have allowed the solution of numerically difficult physical problems in short time-scales, with several publications as a result. My experience using high-performance computing methods informs me of the utility of such methods to both academia and industry.

    During my time in the graduate programme at Okinawa Institute of Science \& Technology Graduate University (OIST), I have gained a variety of experience in many areas. As a formal requirement, one must undertake coursework both inside and outside the area of proposed specialisation. I have spent time working with cryo-electron microscopy, protein crystallization, single particle tomography, 3D molecular reconstruction, fluid mechanics, and optics, outside of my field of expertise, cold atomic systems. Given my diverse background, having spent some time in industry before pursuing PhD study, my resulting work at OIST, as well as my experiences to date, I may offer a unique perspective to a variety of problems.

    Outside of my recent academic pursuits, I have also been elected to Chair of the Student Assembly and held this position for the full length of term, 2015-2016. Through hard work, the Student Council and I were able to directly influence many new rules and policies of this growing university.

If you would like to know more, please do not hesitate to get in touch.
}

\makeletterclosing
\end{document}
