\documentclass[12pt,a4paper,unicode]{moderncv}
\moderncvtheme[grey]{classic}                % oldstyle banking casual classic
\usepackage[utf8]{inputenc}
\usepackage[scale=0.80]{geometry}
\setlength{\hintscolumnwidth}{3cm}

%\hypersetup{colorlinks,breaklinks,urlcolor=linkcolour, linkcolor=linkcolour}

\firstname{Lee James}
\familyname{O'Riordan}
%\title{BSc(Hons), AMInstP}
\address{Quantum Systems Unit, Okinawa Institute of Science and Technology Graduate University}{1919-1 Tancha, Onna-son, Okinawa, Japan 904-0495}
\mobile{+81 (0)80 6498 4323}
\email{lee.oriordan@oist.jp}
\social[linkedin]{loriordan}
\social[github]{mlxd}

\makeatletter

\begin{document}
    \recipient{Microsoft Research}{}
    \date{\today} % Letter date
    \opening{Dear Sir or Madam,}
    \closing{Sincerely,}
    \makelettertitle
{
    I am writing to express interest in working with Microsoft as a postdoctoratal researcher, or researcher. Coming from a physics background, I have strong computational skills that I know will be beneficial and apt for many positions. My area of specialization is in theoretical cold atomic physics, specifically on the dynamics of Bose--Einstein condensates. My PhD work aims to understand the dynamics of perturbations on quantum states with large values of angular momentum. Specifically, in harmonically confined Bose--Einstein condensates, the addition of angular momentum allows for the appearance of topological excitations in the form of quantised vortices. A property of quantised vortices is that the condensate energetically favours many singly charged vortices over multiple charged. As a result, for increasing values of angular momentum a large number of vortices enter the condensate and arrange into a lattice to minimise the system energy.

    %%%%%%%%%%%%%%%%%%%%%%%%%%%%%%%%%%%%%%%%%%%%%%%%%%%%%%%%%%%%%%%%%%%%%

    Understanding the behaviour of Bose--Einstein condensate vortex lattices can help with engineering quantum states for future technologies. Ideally, they can be used for long-term memory storage in quantum computing systems as they are topologically protected and are very robust. They also allow the study of quantum mechanical effects on mesoscopic scales, and the inherent periodicity makes them a great tool for simulating condensed matter physics. Furthermore, perturbed vortex lattices can be used to investigate turbulent, and possibly chaotic, quantum behaviour. While turbulent classical systems are notoriously hard to understand and control, quantum turbulence is thought to offer a more controllable route due to the quantisation condition of the circulation. It is therefore of large interest to develop new tools for manipulating and engineering specific states or rotating condensates.

    In my PhD thesis work I concentrate on two types of perturbations to the equilibrium state of the condensate system: i) the modification of the phonon spectrum of the condensate which does not influence the angular momentum, in particular through the use of a kicked optical potential; ii) the direct control of the topological excitations, and hence the angular momentum, which is performed with direct phase engineering of the wavefunction. Both methods are experimentally realisable using currently available state-of-the-art experimental control techniques.

    %%%%%%%%%%%%%%%%%%%%%%%%%%%%%%%%%%%%%%%%%%%%%%%%%%%%%%%%%%%%%%%%%%%%%

    To better understand these systems I have worked extensively on developing high-performance simulation software using CUDA. The resulting software methods have been independently performance tested, and have been shown to outperform well-established and mature software in C/C++ and \textsc{MATLAB}. If requested, I will be happy to share these results with you. The performance of these algorithms have allowed for the solution of otherwise numerically difficult system parameters in short time-scales, with several publications as a result.

    During my time in the graduate programme at Okinawa Institute of Science \& Technology Graduate University (OIST), I have also gained a variety of experience in many areas outside my specialisation. As a formal requirement, one must undertake coursework both inside and outside the area of proposed specialisation. I have spent time working with cryo-electron microscopy, protein crystallization, single particle tomography, 3D molecular reconstruction, fluid mechanics, and optics. Given my diverse background, having spent some time in industry before pursuing PhD study, my resulting work at OIST, as well as my experiences to date, I may offer a unique perspective to a variety of problems.

    While I have listed my work to date, my interests for future work are not limited to quantum technologies. I have a large interest in the use of high-performance numerical methods to solving real-world problems, as well as those for further advancing quantum technologies. The use of machine learning techniques are becoming more widespread in many disciplines, and I am curious to understand their application to problems of physical systems. Developing methods and techniques using high-performance numerical technologies such as GPU computing and Intel Xeon Phi's can potentially allow for an interesting results in many emerging fields. While I have mentioned these as examples, I am keen to undertake work on interesting problems, and expand my set of skills and experiences.

    %Outside of my recent academic pursuits, I have also been elected to Chair of the Student Assembly and held this position for the full length of term, 2015-2016. Through hard work, the Student Council and I were able to directly influence many new rules and policies of this growing university.

    If you find that I may be a suitable for a position, I would be more than happy to better discuss my possible contributions in person, or via Skype. Please do not hesitate to get in touch for further information.
}

\makeletterclosing
\end{document}
