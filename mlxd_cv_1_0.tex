\documentclass[11pt,a4paper,unicode]{moderncv}

\moderncvtheme[grey]{classic}
%\moderncvtheme[green]{classic}%idem

%\usepackage{fontspec}
%\defaultfontfeatures{Mapping=tex-text,Scale=MatchLowercase}
%\setmainfont{Lusitana-Regular}
%\setmainfont{Palatino}
%\setmonofont{FiraCode-Medium}

\usepackage[utf8]{inputenc}

\usepackage[scale=0.89]{geometry}
\setlength{\hintscolumnwidth}{2.5cm}			% if you want to change the width of the column with the dates
%\AtBeginDocument{\setlength{\maketitlenamewidth}{6cm}}  % only for the classic theme, if you want to change the width of your name placeholder (to leave more space for your address details
%\AtBeginDocument{\recomputelengths}                     % required when changes are made to page layout lengths

\usepackage{hyperref}
\definecolor{linkcolour}{rgb}{0,0.2,0.6}
\hypersetup{colorlinks,breaklinks,urlcolor=linkcolour, linkcolor=linkcolour}

\firstname{Lee James}
\familyname{O'Riordan}
\title{BSc (Hons), PhD}               % optional, remove the line if not wanted
%\address{Quantum Systems Unit, Okinawa Institute of Science and Technology Graduate University}{1919-1 Tancha, Onna-son, Okinawa, Japan 904-0495}   % optional, remove the line if not wanted
\mobile{+353 (0)85 7869457}
%\phone{+81 (0)98 966 1588}
\email{loriordan@gmail.com}
\extrainfo{\href{http://mlxd.github.io}{Blog: https://mlxd.github.io}}
\social[linkedin]{loriordan}
\social[github]{mlxd}

\makeatletter
\renewcommand*{\bibliographyitemlabel}{\@biblabel{\arabic{enumiv}}}
%\renewcommand{\rmdefault}{ptm}
\renewcommand{\rmdefault}{ppl} %***
%\renewcommand{\rmdefault}{pnc} %**
%\renewcommand{\rmdefault}{pbk}
%\renewcommand{\rmdefault}{pzc}
\renewcommand{\rmdefault}{put}
\makeatother

%\nopagenumbers{}                             % uncomment to suppress automatic page numbering for CVs longer than one page
%----------------------------------------------------------------------------------
%            content
%----------------------------------------------------------------------------------
\begin{document}
\maketitle
\vspace*{-30pt}

\section{Personal statement}
\cvline{}{I have developed research and production codes for various topics such as data analysis, high performance computing enabled simulations of quantum systems, tomographic image processing, as well as industry applications. Through these topics I have gained experience in the computational modeling, analytical and numerical solution of problems, as well as project management and resource allocation to enable this. I have worked with C/C++, CUDA, Python, MPI, OpenMP, \textsc{MATLAB}, Julia, Mathematica, Shell (Bash/Zsh), \LaTeX, and Java. I am currently gaining
experience with Haskell, as well as the various quantum software toolkits that have recently become available, such as Microsoft QDK, IBM Qiskit, Rigetti Forest SDK, to name a few.}

\section{Education}
\cventry{2012--2017}{PhD in Science (Physics)}{Quantum Systems Unit, Okinawa Institute of Science and Technology Graduate University}{Japan}{}{Thesis: \emph{``Non-equilibrium vortex dynamics in rapidly rotating Bose--Einstein condensates''}}
%\cventry{2011--2012}{PhD in Physics}{Ultracold quantum gases group, University College Cork}{Ireland}{}{Transferred to Okinawa Institute of Science and Technology Graduate University, Japan after completing 1st academic year.}
\cventry{2006--2010}{BSc (Hons) in Physics with Computing}{Waterford Institute of Technology}{Ireland}{}{Graudated with first class honours (1.1).}%{Thesis: \emph{``Evaluating magnetic susceptibility in Heisenberg chains using OpenCL implementations of Monte Carlo methods.''}} %{Thesis: \emph{``Evaluating magnetic susceptibility in Heisenberg chains using OpenCL implementations of Monte Carlo methods''}.}
%\cventry{2000--2005}{Leaving Certificate}{Ardscoil na mBraithre Clonmel}{Ireland}{}{}
%\cventry{2008}{Computer repair technician}{Computer Centre}{Clonmel, Ireland}{}{Routine analysis and repair of computer hardware \& software issues.}
%\cventry{2006--2008}{Session musician \& instructor}{\textit{Teaching \& performing}}{Various locations}{Ireland}{Guitarist \& music instructor.}
%\cventry{2005}{Sales assistant}{Dunnes Stores}{Clonmel, Ireland}{}{General assistant with store upkeep and operations.}

\section{Work experience}
\cventry{2017--2018}{Exascale Crystallographic Computation Postdoctoral Fellow}{}{Lawrence Berkeley National Lab}{United States}{Researched real-time computational analysis of X-ray free electron laser crystallographic data. Worked on extending existing \href{https://github.com/cctbx}{CCTBX} and \href{https://github.com/dials}{DIALS} software suites to work with Intel Xeon Phi processors on NERSC's Cori supercomputer. Example work available at \href{https://github.com/exafel}{ExaFEL}.}
\cventry{2011--2012}{Research assistant}{}{University College Cork}{Cork, Ireland}{Researching theory of ultracold quantum systems. Organised, guided and demonstrated undergraduate laboratory sessions in physics experiments.}
\cventry{2010--2011}{Software developer}{}{IBM Ireland}{Dublin, Ireland}{Developed server-side applications for use with WebSphere Portal platform.\newline{}
Experience using Java, XML, XSLT, Ant, Shell scripting, JDBC, Python, C.\newline{}
Deployed and managed WebSphere Portal, Lotus Sametime, and Lotus Connections software stacks.}
\cventry{2009}{Product engineer (intern)}{}{Analog Devices BV}{Limerick, Ireland}{Member of semiconductor device yield improvement team.\newline{}
Analysed and implemented strategies for device failure analysis.\newline{}
Developed software for die yield analysis and characterisation.\newline{}
}

\section{Software development experience}
\cvline{GPUE}{Architect \& developer of GPUE: GPU-enabled Gross--Pitaevskii equation solver; a pseudospectral linear and nonlinear partial differential equation solver and simulation tool. Languages: CUDA, C/C++, Python, \textsc{MATLAB}, Shell. Available at \href{http://github.com/gpue-group/gpue}{gpue-group/gpue}.}
\cvline{cctbx}{Computational crystallographic toolbox. Technologies: Python, C++, Boost.Python, OpenMP, MPI. Available at \href{http://github.com/cctbx-project/cctbx}{cctbx-project/cctbx}.}
\cvline{DIALS}{DIALS X-ray diffraction analysis software. Technologies: Python, C++, Boost.Python. Available at \href{http://github.com/dials/dials}{dials/dials}.}
\cvline{ExaFEL Project}{Exascale Free Electron Laser project: exascale capable extensions to cctbx and DIALS projects. Technologies: Python, C++, Boost.Python, MPI, OpenMP, Bash. Available at \href{http://github.com/exafel}{exafel}.}

\section{University service}
\cventry{2017-2020}{Board member}{Board of Councilors}{OIST, Japan}{}{}{}
\cventry{2015-2016}{Chair}{Student Council}{OIST, Japan}{}{Elected by student body. Facilitated regular meetings between the student body and faculty. Improved institutional policies and conditions for students. Developed and founded mentoring program for incoming students.}{}
\cventry{2014-}{HPC advanced users group representative}{Quantum Systems Unit}{OIST, Japan}{}{Research unit representative to the OIST HPC and scientific computing team. Trained group members on the use of HPC software and advanced programming techniques}{}
\cventry{2013-2016}{President}{Music Club}{OIST, Japan}{}{Founded and run Music club at OIST. Practiced, trained, and performed with many members on a regular basis.}{}
\cventry{2013-2016}{ITSSC member}{OIST}{}{}{Member of university task force for updating IT policies and compliance.}{}
\cventry{2013-2015}{Lecturer, mentor, speaker}{Quantum Systems Unit}{OIST, Japan}{}{Delivered lectures to local schools on quantum physics. Designed and delivered a team-based Scratch programming language tutorial for children aged 8-16. Mentored several intern students and trained them on numerical computing during PhD program.}{}

%\section{Research interests}
%\cvline{}{I currently research theoretical and numerical solutions of ultracold quantum gas systems. I primarily work on numerical simulations of Bose--Einstein condensate dynamics, with an emphasis on solutions using GPU computing (CUDA). I have worked on full 3D Schr\"odinger solutions of systems for realising matter-wave spatial adiabatic passage, 2D order/disorder transitions in vortex lattice systems, as well as kicked lattice dynamics as a route to quantum chaos. I am interested in investigating quantum turbulence in two-dimensional condensates using controllable topological manipulations, as well as attaining an understanding of quantum chaos in such systems. I aim to develop tools for investigating strongly-correlated behaviour of bosons with large angular momentum for quantum Hall physics, as well as the use of artificial gauge fields for creating simulators of condensed matter systems using cold-atoms. I am also interested in applying state-of-the-art computing techniques in the pursuit of solutions to complex problems. I have developed and open-sourced all simulation code on 2D condensate dynamics (GPUE), and I am keen to continuously update and further the applicability of the code with my collaborators to solve more general problems.}%I have gained substantial experience in computational modeling, analytical, and numerical solution of problems using C/C++, CUDA, \textsc{MATLAB}, Mathematica, Python, and pencil \& paper.

%\pagebreak
\section{Publications}
%\cvline{2018}{\textit{``GPUE: Graphics Processing Unit Gross-Pitaevskii Equation solver''}{ J.~R.~Schloss, L.~J.~O'Riordan } {Journal of Open Source Software, 2018}. }


%\cvline{2017}{\textit{``Controlling vortex rings in Bose-Einstein condensates using artificial gauge fields'', }{J.~Schloss, R.~Sachdeva, L.~J.~O'Riordan, Th.~Busch.} \href{http://journals.aps.org/pra/abstract/10.1103/PhysRevA.94.053603}{Phys. Rev. A 94, 053603 (2016)}. DOI: {10.1103/PhysRevA.94.053603}}

\cvline{2018}{
    \textbf{{Improving signal strength in serial crystallography with DIALS geometry refinement}},
    {A.~S.~Brewster, D.~G.~Waterman, J.~M.~Parkhurst, R.~J.~Gildea, I.~D.~Young, L.~J.~O'Riordan, J.~Yano, G.~Winter, G.~Evans, N.~K.~Sauter},
    {Acta Cryst. D., \textbf{74}, 877-894}. DOI: {\href{https://doi.org/10.1107/S2059798318009191}{10.1107/S2059798318009191} }
}

\cvline{2017}{
    \textbf{{Non-equilibrium vortex dynamics in rapidly rotating Bose-Einstein condensates}},
    {L.~J.~O'Riordan, } 
    \href{https://ci.nii.ac.jp/naid/500001054902/}{Okinawa Insitute of Science and Technology Graduate University}. 
    DOI: { \href{https://doi.org/10.15102/1394.00000165}{10.15102/1394.00000165}} 
}

\cvline{2016}{
    \textbf{{Topological defect dynamics of vortex lattices in Bose--Einstein condensates}},
    {L.~J.~O'Riordan, Th.~Busch.} 
    \href{http://journals.aps.org/pra/abstract/10.1103/PhysRevA.94.053603}{Phys. Rev. A 94, 053603}. 
    DOI: { \href{https://doi.org/10.1103/PhysRevA.94.053603}{10.1103/PhysRevA.94.053603} } 
}

\cvline{2016}{
    \textbf{{Moir\'e superlattice structures in kicked Bose-Einstein condensates}},
    {L.~J.~O'Riordan, A.~C.~White, Th.~Busch.} 
    \href{https://journals.aps.org/pra/abstract/10.1103/PhysRevA.93.023609}{Phys. Rev. A 93, 023609}. 
    DOI: { \href{https://doi.org/10.1103/PhysRevA.93.023609}{10.1103/PhysRevA.93.023609}} 
}

\cvline{2013}{
    \textbf{{Coherent transport by adiabatic passage on atom chips}},
    {T.~Morgan, L.~J.~O'Riordan, N.~Crowley, Th.~Busch.} 
   \href{http://journals.aps.org/pra/abstract/10.1103/PhysRevA.88.053618}{Phys. Rev. A 88, 053618}.
    DOI: { \href{https://doi.org/10.1103/PhysRevA.88.053618}{10.1103/PhysRevA.88.053618}} 
}

\section{Presentations}
\cventry{Mar, 2018}{DIALS Workshop 2018} {Talk}{Lawrence Berkeley National Lab}{USA}{Title: ``\emph{Data analysis and development using Jupyter}"}
\cventry{Apr, 2017}{Waterford Institute of Technology} {Seminar}{Waterford}{Ireland}{Title: ``\emph{Ultracold atomic gases}"}
\cventry{Mar, 2016}{WQS 2016} {Conference}{Dunedin}{New Zealand}{Poster presentation: ``\emph{Dynamics of large vortex lattice carrying Bose--Einstein condensates}"}
\cventry{June, 2015}{ICOLS 2015} {Conference}{Sentosa}{Singapore}{Poster presentation: ``\emph{Moiré super-lattice in a kicked rapidly rotating Bose–Einstein condensate}"}
\cventry{Sept, 2014}{Coherent Quantum Dynamics} {Workshop}{Okinawa Institute of Science and Technology Graduate University}{Japan}{Poster presentation: ``\emph{Investigating vortex dynamics in harmonically trapped Bose-Einstein condensates}"}
\cventry{July, 2014}{Non-linear Dynamics, Dynamical Transitions and Instabilities in Classical and Quantum Systems}{Workshop}{ICTP}{Italy}{Poster presentation: ``\emph{Investigating vortex dynamics in harmonically trapped Bose-Einstein condensates}"}
\cventry{April, 2014}{C3QS 2014}{Conference}{Okinawa Institute of Science and Technology Graduate University}{Japan}{Poster presentation: ``\emph{Coherent transport by adiabatic passage on atom chips}"}
\cventry{Sept, 2013}{Manipulation of degenerate quantum gases}{Workshop}{\'{E}cole Pr\'{e}doctorale de Physique des Houches}{France}{Poster presentation: ``\emph{Coherent transport by adiabatic passage on atom chips}"}
\cventry{May, 2013}{C3QS 2013}{Conference}{Okinawa Institute of Science and Technology Graduate University}{Japan}{Poster presentation:  ``\emph{Simulating 3D atomic dynamics using a graphics processor accelerated split-operator method}"}
%\cventry{Sept, 2012}{QuAMP 2012}{Workshop}{Queens University Belfast}{Northern Ireland}{Attendee}
\cventry{Jul, 2012}{QUACS 2012}{Workshop}{University of Nottingham}{UK}{Poster presentation: ``\emph{Simulations of 3D atomic dynamics using a GPU accelerated split-operator method}"}
\cventry{Mar, 2012}{Institute of Physics Ireland Spring Meeting 2012}{Workshop}{Royal College of Surgeons}{Ireland}{Poster presentation: ``\emph{Simulations of 3D atomic dynamics using a GPU accelerated split-operator method}"}
\cventry{Mar, 2012}{DPG Spring Meeting}{University of Stuttgart}{Germany}{Conference}{Oral presentation: ``\emph{Simulations of 3D atomic dynamics using a GPU accelerated split-operator method}"}

%\section{Graduate courses}
%\cventry{2014}{Coherent Quantum Dynamics}{Multiple lecturers}{Okinawa Institute of Science and Technology Graduate University}{Japan}{}
%\cventry{}{School on Non-linear Dynamics, Dynamical Transitions and Instabilities in Classical and Quantum Systems}{Multiple lecturers}{ICTP}{Italy}{}
%\cventry{2013}{Cold-Atoms PreDoc School: Manipulation of degenerate quantum gases}{Multiple lecturers}{\'{E}cole Pr\'{e}doctorale de Physique des Houches}{France}{}
%\cventry{}{Advanced Optics}{Prof. S. NicCormaic}{Okinawa Institute of Science and Technology Graduate University}{Japan}{}
%\cventry{}{Structural Biology}{Prof. U. Skoglund, Prof. F. Samatey}{Okinawa Institute of Science and Technology Graduate University}{Japan}{}
%\cventry{}{Classical Electrodynamics}{Prof. T. Shintake}{Okinawa Institute of Science and Technology Graduate University}{Japan}{}
%\cventry{}{Continuum Mechanics (Fluids)}{Prof. P. Chakraborty}{Okinawa Institute of Science and Technology Graduate University}{Japan}{}
%\cventry{2012}{Phase Space Methods for Quantum Atom Optics}{Prof. B. Dalton}{University College Cork}{Ireland}{}

\section{Competitions \& awards}
\cventry{2014}{Best Customer Interaction}{Kyued-Up Innovation Event}{Okinawa Institute of Science and Technology Graduate University, Japan}{URL: \href{http://pullapproach.com}{http://pullapproach.com}}{
Project title: ``Okinawa Science Discovery Center''}{}
\cventry{2013}{2nd place}{Euraxess Links Japan, Science Slam 2013}{Tokyo Institute of Technology}{Japan}{Talk title: ``Quantum Typhoons''}{}
\cventry{2010}{Student of the Year}{School of Science, Waterford Institute of Technology}{Ireland}{}{}
\cventry{2010}{Best undergraduate project}{Department of Computing, Maths and Physics, Waterford Institute of Technology}{Ireland}{}{}
%\cventry{2010}{Earnshaw Medal Nominee}{Institute of Physics}{Ireland}{}{}
%\cventry{2007}{3rd place}{Robocode: Inter-varsity AI programming competition}{Ireland}{}{}
\iffalse
\section{Reference contact details}
\cvline{Postdoctoral supervisor}{Dr. Nicholas Sauter, Lawrence Berkeley National Lab, California, United States. \newline{}  {tel: +1-510-486-5713,  e-mail:\href{nksauter@lbl.gov}{nksauter@lbl.gov}}}
\cvline{PhD Supervisor}{Prof. Thomas Busch, Okinawa Institute of Science and Technology Graduate University, Japan. \newline{} tel: +81 (0)98 966 1588,  e-mail:\href{thomas.busch@oist.jp}{thomas.busch@oist.jp}}
\cvline{Research colleague}{Dr. Angela White, Okinawa Institute of Science and Technology Graduate University, Japan, and Australian National University, Australia. \newline{}  {e-mail:\href{angela.white@oist.jp}{angela.white@oist.jp}}}
\fi
\begin{center}
\textbf{Please contact me for references.}
\end{center}

\clearpage
\iffalse
\recipient{ICHEC}{National University of Ireland Galway} % Letter recipient
\date{\today} % Letter date
\opening{Dear ICHEC hiring team,} % Opening greeting
\closing{Sincerely,} % Closing phrase
%\enclosure[Attached]{curriculum vit\ae}
%
\makelettertitle
%
{This letter will form part of my application for the position advertised as Research Computational Scientist. Below, I will outline my experience to date, and applicability for this position. I have completed my PhD in 2017, and have followed that up by taking a postdoctoral role allowing me to explore current trends in developing exascale computational software methods, and further develope my computational and data analysis experience.

I am interested in working within the HPC technology realm for my career, as I have developed significant interest in accelerator-enabled computing. During my PhD I worked on simulations of cold atomic physical systems through numerical solution of the non-linear Schr\"odinger equation to study the dynamics of Bose--Einstein condensates (BEC). My most recent published works have been working with the dynamics of quantised vorticity in these BEC systems. To simulate these systems I developed a CUDA-enabled codebase, which I have open-sourced and placed at https://github.com/gpue-group/gpue .  While not all of the work developed on this project is currently open due to continuing research in the area, many of the CUDA kernels, C/C++ code and Python modules are all available, as well as some MATLAB scripts to augment the work. The code was independently tested against other existing packages in the field, and GPUE was found to be the fastest at the at the time of examination (see https://peterwittek.com/gpe-comparison.html for further details). My published works using this GPU code are [Phys. Rev. A 88, 053618 (2013), Phys. Rev. A 93, 023609 (2016), Phys. Rev. A 94, 053603 (2016)], and I expect additional papers to be published in the coming months.

Upon completion of my PhD I accepted a role at Lawrence Berkeley National Lab (LBNL) examining X-ray crystallography data from the LCLS (Linear coherent light source) at SLAC, Stanford. The goal of this project is to enable the processing of many millions of images in realtime, where NERSC’s supercomputer Cori will be used for this processing. The routines also require preparation for the accelerator upgrade to LCLS2 which is expected to generate data at rates several orders of magntiude larger. To date we have successfully ported CCTBX (Computational Crystallography Toolbox), the de-facto standard for use in this field, to NERSC's Cori supercomputer, verified that we process data as it is collected at the detector, and obtain results in a shoter time than previously. We also capably showed that several components of the data analysis problem can scale well with the addition of nodes, and identified others which require further development to improve scalability. Moving forward will require optimisation and streamlining of the processing stages of the software pipeline to allow for higher data rates, as well as optimisation of routines to fully utilise the Intel Knights Landing hardware. While I am no longer directly on this project, I remain an affiliate of LBNL to assist in the short-term with the work, and remain a developer on the open-sourced modules in the project. Technologies used on this project were MPI, OpenMP, C/C++ with Python bindings written in Boost.Python, as well as the Shifter container format developed at NERSC (based on Docker).

Going forward in my career, I believe that my experiences are better suited to enabling users to scale and optimise their applications for use on exotic hardwares (GPU, KNL), as well as achieving optimal algorithmic complexity, scalability and performance using existing technologies such as MPI, OpenMP. I am also interested in modern approaches to computation, such as the recent Julia language, which has been demonstrated as the first non C/C++/Fortran language to break the sustained PetaFLOPs barrier.

I am quite happy to discuss in depth my background and applicability for this position, should you so wish. Thank you in advance, and I look forward to your response.

}
%
\makeletterclosing % Print letter signature
\fi
\end{document}
