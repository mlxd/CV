\documentclass[11pt,a4paper,unicode]{moderncv}

\moderncvtheme[grey]{classic}
%\moderncvtheme[green]{classic}%idem

\usepackage[utf8]{inputenc}

\usepackage[scale=0.86]{geometry}
\setlength{\hintscolumnwidth}{2.5cm}			% if you want to change the width of the column with the dates
%\AtBeginDocument{\setlength{\maketitlenamewidth}{6cm}}  % only for the classic theme, if you want to change the width of your name placeholder (to leave more space for your address details
%\AtBeginDocument{\recomputelengths}                     % required when changes are made to page layout lengths

\usepackage{hyperref}
\definecolor{linkcolour}{rgb}{0,0.2,0.6}
\hypersetup{colorlinks,breaklinks,urlcolor=linkcolour, linkcolor=linkcolour}

\firstname{Lee James}
\familyname{O'Riordan}
%\title{BSc(Hons), AMInstP}               % optional, remove the line if not wanted
%\address{Quantum Systems Unit, Okinawa Institute of Science and Technology Graduate University}{1919-1 Tancha, Onna-son, Okinawa, Japan 904-0495}   % optional, remove the line if not wanted
\mobile{+81 (0)80 6498 4323}
%\phone{+81 (0)98 966 1588}
\email{lee.oriordan@oist.jp}
\extrainfo{\href{http://groups.oist.jp/qsu}{http://groups.oist.jp/qsu}}
\social[linkedin]{loriordan}
\social[github]{mlxd}


\makeatletter
\renewcommand*{\bibliographyitemlabel}{\@biblabel{\arabic{enumiv}}}
\makeatother

%\nopagenumbers{}                             % uncomment to suppress automatic page numbering for CVs longer than one page
%----------------------------------------------------------------------------------
%            content
%----------------------------------------------------------------------------------
\begin{document}
\maketitle
\vspace*{-30pt}
\section{Education}
\cventry{2012--Present}{PhD in Science (Physics)}{Quantum Systems Unit, Okinawa Institute of Science and Technology Graduate University}{Japan}{Expected completion: March 2017}{Thesis: \emph{``Non-equilibrium vortex dynamics in rapidly rotating Bose--Einstein condensates''}}
\cventry{2011--2012}{PhD in Physics}{Ultracold quantum gases group, University College Cork}{Ireland}{}{Transferred to Okinawa Institute of Science and Technology Graduate University, Japan after completing 1st academic year.}
\cventry{2006--2010}{BSc (Hons) in Physics with Computing}{Waterford Institute of Technology}{Ireland}{Graduated with first class honours (1.1).}{Thesis: \emph{``Evaluating magnetic susceptibility in Heisenberg chains using OpenCL implementations of Monte Carlo methods.''}}%{Thesis: \emph{``Evaluating magnetic susceptibility in Heisenberg chains using OpenCL implementations of Monte Carlo methods''}.}
%\cventry{2000--2005}{Leaving Certificate}{Ardscoil na mBraithre Clonmel}{Ireland}{}{}

\section{Experience}
\cventry{2013,2015}{MATLAB instructor}{\textit{Teaching}}{Okinawa Institute of Science and Technology Graduate University}{Japan}{Delivered lectures on using \textsc{MATLAB} as a research computing environment with emphasis on acceleration methods for numerical problems. The target audience members were PhD students, postdoctoral researchers and staff.}
\cventry{2011--2012}{Laboratory teaching assistant}{\textit{Teaching}}{University College Cork}{Cork, Ireland}{Organised, guided and demonstrated undergraduate laboratory sessions in physics experiments.}
\cventry{2010--2011}{Software engineer}{\textit{Work}}{IBM Ireland}{Dublin, Ireland}{Developed server-side applications for use with WebSphere Portal platform.\newline{}
Experience using Java, XML, XSLT, Ant, Shell scripting, JDBC, Python, C.\newline{}
Deployed and managed WebSphere Portal, Lotus Sametime, and Lotus Connections software stacks.}

\cventry{2009}{Product engineer (intern)}{\textit{Work}}{Analog Devices BV}{Limerick, Ireland}{Member of semiconductor device yield improvement team.\newline{}
Analysed and implemented strategies for device failure analysis.\newline{}
Developed software for die yield analysis and characterisation.\newline{}
}


%\cventry{2008}{Computer repair technician}{Computer Centre}{Clonmel, Ireland}{}{Routine analysis and repair of computer hardware \& software issues.}
%\cventry{2006--2008}{Session musician \& instructor}{\textit{Teaching \& performing}}{Various locations}{Ireland}{Guitarist \& music instructor.}
%\cventry{2005}{Sales assistant}{Dunnes Stores}{Clonmel, Ireland}{}{General assistant with store upkeep and operations.}


%\section{Bachelor thesis}
%\cvline{Title}{\emph{Evaluating magnetic susceptibility in Heisenberg chains using OpenCL implementations of Monte Carlo methods}}
%\cvline{Supervisors}{Dr. Kieran Murphy}
%\cvline{Description}{\small Computational project using graphics processors for accelerated evaluation; Provided means for calculating values of susceptibility given a specific chain length of atoms subject to applied fields; implementation used Monte Carlo integration methods programmed in OpenCL \& C; Results
%showed excellent agreement with zero-field analytical case, much higher performance than other implementations for numerical evaluation with applied field.}

%\section{Research interests}
%\cvline{}{I currently research theoretical and numerical solutions of ultracold quantum gas systems. I primarily work on numerical simulations of Bose--Einstein condensate dynamics, with an emphasis on solutions using GPU computing (CUDA). I have worked on full 3D Schr\"odinger solutions of systems for realising matter-wave spatial adiabatic passage, 2D order/disorder transitions in vortex lattice systems, as well as kicked lattice dynamics as a route to quantum chaos. I am interested in investigating quantum turbulence in two-dimensional condensates using controllable topological manipulations, as well as attaining an understanding of quantum chaos in such systems. I aim to develop tools for investigating strongly-correlated behaviour of bosons with large angular momentum for quantum Hall physics, as well as the use of artificial gauge fields for creating simulators of condensed matter systems using cold-atoms. I am also interested in applying state-of-the-art computing techniques in the pursuit of solutions to complex problems. I have developed and open-sourced all simulation code on 2D condensate dynamics (GPUE), and I am keen to continuously update and further the applicability of the code with my collaborators to solve more general problems.}%I have gained substantial experience in computational modeling, analytical, and numerical solution of problems using C/C++, CUDA, \textsc{MATLAB}, Mathematica, Python, and pencil \& paper.

%\pagebreak
\section{Publications}
\cvline{2016}{\textit{``Topological defect dynamics of vortex lattices in Bose--Einstein condensates'', }{L.~J.~O'Riordan, Th.~Busch.} \href{http://journals.aps.org/pra/abstract/10.1103/PhysRevA.94.053603}{Phys. Rev. A 94, 053603 (2016)}. DOI: {10.1103/PhysRevA.94.053603}}
\cvline{2016}{\textit{``Moir\'e superlattice structures in kicked Bose-Einstein condensates'', }{L.~J.~O'Riordan, A.~C.~White, Th.~Busch.} \href{https://journals.aps.org/pra/abstract/10.1103/PhysRevA.93.023609}{Phys. Rev. A 93, 023609 (2016)}. DOI: {10.1103/PhysRevA.93.023609}}
\cvline{2013}{\textit{``Coherent transport by adiabatic passage on atom chips'', }{T.~Morgan, L.~J.~O'Riordan, N.~Crowley, Th.~Busch,} \href{http://journals.aps.org/pra/abstract/10.1103/PhysRevA.88.053618}{Phys. Rev. A 88, 053618 (2013)}. DOI: {10.1103/PhysRevA.88.053618}}



\section{Conferences \& workshops}
\cventry{Mar, 2016}{WQS 2016} {Conference}{Dunedin}{New Zealand}{Poster presentation: ``\textit{Dynamics of large vortex lattice carrying Bose--Einstein condensates}"}
\cventry{June, 2015}{ICOLS 2015} {Conference}{Sentosa}{Singapore}{Poster presentation: ``\textit{Moiré super-lattice in a kicked rapidly rotating Bose–Einstein condensate}"}
\cventry{Sept, 2014}{Coherent Quantum Dynamics} {Workshop}{Okinawa Institute of Science and Technology Graduate University}{Japan}{Poster presentation: ``\textit{Investigating vortex dynamics in harmonically trapped Bose-Einstein condensates}"}
\cventry{July, 2014}{Non-linear Dynamics, Dynamical Transitions and Instabilities in Classical and Quantum Systems}{Workshop}{ICTP}{Italy}{Poster presentation: ``\textit{Investigating vortex dynamics in harmonically trapped Bose-Einstein condensates}"}
\cventry{April, 2014}{C3QS 2014}{Conference}{Okinawa Institute of Science and Technology Graduate University}{Japan}{Poster presentation: ``\textit{Coherent transport by adiabatic passage on atom chips}"}
\cventry{Sept, 2013}{Manipulation of degenerate quantum gases}{Workshop}{\'{E}cole Pr\'{e}doctorale de Physique des Houches}{France}{Poster presentation: ``\textit{Coherent transport by adiabatic passage on atom chips}"}
\cventry{May, 2013}{C3QS 2013}{Conference}{Okinawa Institute of Science and Technology Graduate University}{Japan}{Poster presentation:  ``\textit{Simulating 3D atomic dynamics using a graphics processor accelerated split-operator method}"}
%\cventry{Sept, 2012}{QuAMP 2012}{Workshop}{Queens University Belfast}{Northern Ireland}{Attendee}
\cventry{Jul, 2012}{QUACS 2012}{Workshop}{University of Nottingham}{UK}{Poster presentation: ``\textit{Simulations of 3D atomic dynamics using a GPU accelerated split-operator method}"}
\cventry{Mar, 2012}{Institute of Physics Ireland Spring Meeting 2012}{Workshop}{Royal College of Surgeons}{Ireland}{Poster presentation: ``\textit{Simulations of 3D atomic dynamics using a GPU accelerated split-operator method}"}
\cventry{Mar, 2012}{DPG Spring Meeting}{University of Stuttgart}{Germany}{Conference}{Oral presentation: ``\textit{Simulations of 3D atomic dynamics using a GPU accelerated split-operator method}"}

%\section{Graduate courses}
%\cventry{2014}{Coherent Quantum Dynamics}{Multiple lecturers}{Okinawa Institute of Science and Technology Graduate University}{Japan}{}
%\cventry{}{School on Non-linear Dynamics, Dynamical Transitions and Instabilities in Classical and Quantum Systems}{Multiple lecturers}{ICTP}{Italy}{}
%\cventry{2013}{Cold-Atoms PreDoc School: Manipulation of degenerate quantum gases}{Multiple lecturers}{\'{E}cole Pr\'{e}doctorale de Physique des Houches}{France}{}
%\cventry{}{Advanced Optics}{Prof. S. NicCormaic}{Okinawa Institute of Science and Technology Graduate University}{Japan}{}
%\cventry{}{Structural Biology}{Prof. U. Skoglund, Prof. F. Samatey}{Okinawa Institute of Science and Technology Graduate University}{Japan}{}
%\cventry{}{Classical Electrodynamics}{Prof. T. Shintake}{Okinawa Institute of Science and Technology Graduate University}{Japan}{}
%\cventry{}{Continuum Mechanics (Fluids)}{Prof. P. Chakraborty}{Okinawa Institute of Science and Technology Graduate University}{Japan}{}
%\cventry{2012}{Phase Space Methods for Quantum Atom Optics}{Prof. B. Dalton}{University College Cork}{Ireland}{}

\section{Programming experience}
\cvline{}{I have extensive experience in programming with many different languages. I have developed research codes for various topics such as data analysis, high performance computing enabled simulations of quantum systems, and tomographic image processing, and have gained substantial experience in computational modeling, analytical, and numerical solution of problems. Currently, I have worked with C/C++, CUDA, Python, \textsc{MATLAB}, Julia, Mathematica, Shell (Bash/Zsh), \LaTeX, and Java. Listed below are a subset of projects that I have written, and are available on GitHub.}
\cvline{GPUE}{Architect \& developer of GPUE: GPU-enabled Gross--Pitaevskii equation solver; a pseudospectral linear and nonlinear partial differential equation solver and simulation tool. Languages: CUDA, C/C++, Python, \textsc{MATLAB}, Shell. Available at \href{http://github.com/mlxd/gpue}{mlxd/gpue}.}
\cvline{GraphIt}{Graph generation library for 2D spatial data. Language: C++11. Available at \href{http://github.com/mlxd/GraphIt}{mlxd/GraphIt}. }
\cvline{hexacorr}{Correlation functions for 2D spatial data. Used extensively in above publications. Language: \textsc{MATLAB}. Available at \href{http://github.com/mlxd/hexacorr}{mlxd/hexacorr}.}
\cvline{Sparsey.jl}{Finite difference sparse matrix numerical diagonalizer of the quantum harmonic oscillator Hamiltonian. Preliminary work. Language: Julia. Available at \href{http://github.com/mlxd/sparsey.jl}{mlxd/sparsey.jl}.}

\section{University service}
\cventry{2015-2016}{Chair}{Student Council}{OIST, Japan}{}{Elected by student body. Facilitated regular meetings between the student body and faculty. Improved institutional policies and conditions for students. Developed and founded mentoring program for incoming students.}{}
\cventry{2014-}{HPC advanced users group representative}{Quantum Systems Unit}{OIST, Japan}{}{Research unit representative to the OIST HPC and scientific computing team. Trained group members on the use of HPC software and advanced programming techniques}{}
\cventry{2013-2016}{President}{Music Club}{OIST, Japan}{}{Founded and run Music club at OIST. Practiced, trained, and performed with many members on a regular basis.}{}
\cventry{2013-2016}{ITSSC member}{OIST}{}{}{Member of university task force for updating IT policies and compliance.}{}
\cventry{2013-2015}{Lecturer, mentor, speaker}{Quantum Systems Unit}{OIST, Japan}{}{Delivered lectures to local schools on quantum physics. Designed and delivered a team-based Scratch programming language tutorial for children aged 8-16. Mentored several intern students and trained them on numerical computing during PhD program.}{}

\section{Competitions \& awards}
\cventry{2014}{Best Customer Interaction}{Kyued-Up Innovation Event}{Okinawa Institute of Science and Technology Graduate University, Japan}{URL: \href{http://pullapproach.com}{http://pullapproach.com}}{
Project title: ``Okinawa Science Discovery Center''}{}
\cventry{2013}{2nd place}{Euraxess Links Japan, Science Slam 2013}{Tokyo Institute of Technology}{Japan}{Talk title: ``Quantum Typhoons''}{}
\cventry{2010}{Student of the Year}{School of Science, Waterford Institute of Technology}{Ireland}{}{}
\cventry{2010}{Best undergraduate project}{Department of Computing, Maths and Physics, Waterford Institute of Technology}{Ireland}{}{}
%\cventry{2010}{Earnshaw Medal Nominee}{Institute of Physics}{Ireland}{}{}
%\cventry{2007}{3rd place}{Robocode: Inter-varsity AI programming competition}{Ireland}{}{}




\iffalse
\section{References}
\cvline{PhD Supervisor}{Prof. Thomas Busch, Okinawa Institute of Science and Technology Graduate University, Okinawa, Japan. \newline{} tel: +81 (0)98 966 1588,  e-mail:\href{thomas.busch@oist.jp}{thomas.busch@oist.jp}}
\cvline{PhD academic mentor}{Prof. Nic Shannon, Okinawa Institute of Science and Technology Graduate University, Okinawa, Japan. \newline{}  {e-mail:\href{nic.shannon@oist.jp}{nic.shannon@oist.jp}}}
\cvline{Postdoctoral researcher}{Dr. Angela White, Okinawa Institute of Science and Technology Graduate University, Okinawa, Japan. \newline{}  {e-mail:\href{angela.white@oist.jp}{angela.white@oist.jp}}}
\fi

\begin{center}
\textbf{Please contact me for references.}
\end{center}



%\clearpage

%\recipient{École Prédoctorale de Physique des Houches}{Centre de Physique des Houches, Côte des Chavants, F-74310 Les Houches, France.} % Letter recipient
%\date{\today} % Letter date
%\opening{Dear Sir or Madam,} % Opening greeting
%\closing{Sincerely,} % Closing phrase
%\enclosure[Attached]{curriculum vit\ae}
%
%\makelettertitle
%
%{I am enquiring about the advertised position ``'' you have advterised on LinkedIn.
%The topics covered on this year's programme are very relevant to my areas of study as part of my PhD thesis.
%During my first year of PhD studies at University College Cork (UCC) under Thomas Busch I have worked on coherent atomic control using atom-chip structures.
%Employing the matter-wave analogue of STIRAP, coherent tunneling via adiabatic passage (CTAP) as outlined by Eckert et al (doi:10.1016/j.optcom.2006.02.056), and extending the work carried out by
%O'Sullivan et al. (doi:10.1088/0031-8949/2010/T140/014029) on 2D trapping potentials, a fully experimentally realisable 3D system was devised.  Simulations of our proposed system showed verification of the CTAP process, wherein we were able to show transfer fidelities of 99.8\% for counter-intuitive waveguide arrangements, and verification of the difficulties with direct tunneling via intuitive arrangements.
%I presented this work at DPG 2012, with the final version of the resulting paper currently being finalised.
%Upon completion of this body of work, I have commenced work on analysing vortex dynamics in Bose-Einstein condensates. I wish to investigate the dynamics of a vortex lattice structure wherein, given sufficient rotation, the vortices will form an Abrikosov lattice pattern in the ground-state. Given careful choice of lattice spacings, periodic pulsing of an optical lattice pattern may yield dynamics akin to that of a delta-kicked harmonic oscillator, and allow for a means of visualising quantum turbulent or chaotic behaviour. As part of this work I have
%developed a program for performing integration of the time dependent Gross-Pitaevskii equation which outperforms equivalent programs in Fortran, C and Matlab for achieving results in short timescales.
%More recently, having transferred to the graduate programme at Okinawa Institute of Science \& Technology (OIST), I have gained more variety in my studies. As a formal requirement, one must take coursework both inside and outside the area of proposed specialisation. I have spent time working in a cryo-electron microscopy biology group,
%and taken courses on protein crystallization and in single particle tomography and 3D molecular reconstruction. Given my diverse background,
%having spent some time in industry before pursuing PhD study, and resulting work carried out as part of the OIST PhD programme, I may offer a unique
%perspective on many topics. Having a background in high-performance computing I believe is also beneficial, and I may offer insight to many others
%who wish to perform simulations of complex physical systems in reasonable timescales. Lastly, this school would also offer the ability for me to build my academic network with the possibility of developing future collaborative efforts with other such attendees. It is for these reasons that I wish to attend this summer school.
%}
%
%I look forward to discussing my qualifications, and can be reached at \href{loriordan@gmail.com}{\textcolor{blue}{loriordan@gmail.com}}.
%\makeletterclosing % Print letter signature
\end{document}
