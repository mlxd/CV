\documentclass[10pt,a4paper,unicode]{moderncv}

\moderncvtheme[grey]{classic}                % oldstyle banking casual classic

\usepackage[utf8]{inputenc}

\usepackage[scale=0.88]{geometry}
\setlength{\hintscolumnwidth}{2.5cm}

\usepackage{hyperref}								% to use hyperlinks
\definecolor{linkcolour}{rgb}{0,0.2,0.6}			% hyperlinks setup
\hypersetup{colorlinks,breaklinks,urlcolor=linkcolour, linkcolor=linkcolour}

\firstname{Lee James}
\familyname{O'Riordan}
\title{PhD}               % optional, remove the line if not wanted
%\address{Quantum Systems Unit, Okinawa Institute of Science and Technology Graduate University}{1919-1 Tancha, Onna-son, Okinawa, Japan 904-0495}   % optional, remove the line if not wanted
\mobile{+1 (510) 495 4984}
%\phone{+81 (0)98 966 1588}
\email{loriordan@lbl.gov}
\extrainfo{\href{http://mlxd.github.io}{http://mlxd.github.io}}
\social[linkedin]{loriordan}
\social[github]{mlxd}

\makeatletter
\renewcommand*{\bibliographyitemlabel}{\@biblabel{\arabic{enumiv}}}
\makeatother

%\nopagenumbers{}

%----------------------------------------------------------------------------------
%            content
%----------------------------------------------------------------------------------
\begin{document}
\maketitle
\vspace*{-30pt}

\section{Personal statement}
\cvline{}{I have developed research software for various topics such as high-dimensional data analysis, high performance computing simulations of quantum systems, and analysis of high-volume X-ray crystallographic data. Experienced with C/C++, CUDA, Python, \textsc{MATLAB}, OpenCL, Julia, Mathematica, Shell (Bash/Zsh), \LaTeX, and Java. I am interested in quantum technologies and computing, HPC and accelerated computational methods, and using the latest computational approaches to solve applied to real-world problems. Listed below are a subset of projects that I have contributed to:}
\cvline{GPUE}{CUDA-enabled Gross--Pitaevskii equation solver; a pseudo-spectral linear and nonlinear partial differential equation solver and simulation tool for superfluid dynamics. Technologies: CUDA, C/C++, Python, \textsc{MATLAB}, Shell. \href{http://github.com/gpue-group/gpue}{http://github.com/gpue-group/gpue}}
\cvline{ExaFEL}{Exascale Computing Project (ECP) research to develop exascale-capable extensions to cctbx.xfel (\href{http://github.com/cctbx}{http://github.com/cctbx}) and DIALS (\href{http://github.com/dials}{http://github.com/dials}) computational crystallography software for molecular biology and materials research. Technologies: C++, Python, Boost.Python \href{http://github.com/exafel}{http://github.com/exafel}.}
\cvline{}{For my organisational and advisory experiences during my PhD I was elected to be a member of the OIST Board of Councilors at Okinawa Institute of Science and Technology, Japan, and will hold this position until 2020.}

\section{Education}
\cventry{2012--2017}{PhD in Science (Physics)}{Quantum Systems Unit, Okinawa Institute of Science and Technology Graduate University (OIST)}{Japan}{}{}
\cventry{2011--2012}{PhD in Physics (partial)}{Ultracold Quantum Gases group, University College Cork (UCC)}{Ireland}{}{Transferred to OIST, Japan after completing first year.}
\cventry{2006--2010}{BSc (Hons) in Physics with Computing}{Waterford Institute of Technology (WIT)}{Ireland}{First class honours}{}

\section{Experience}
\cventry{2017--Present}{Exascale Crystallographic Computation Postdoctoral Fellow}{}{Lawrence Berkeley National Laboratory}{USA}{Developing exascale extensions to ``cctbx.xfel'' and ``DIALS'' crystallographic analysis software using Intel Xeon Phi (Knights Landing) on NERSC's Cori supercomputer.}{}
\cventry{2012--2017}{Research assistant (PhD)}{}{OIST, Japan}{}{Researching theory of cold atomic quantum systems, specifically Bose--Einstein condensates. Experience in linear algebra, multivariate calculus, differential equations, analysis, probability theory, and numerical algorithms.}{}
%\cventry{2013,2015}{ \textsc{MATLAB} instructor}{\textit{Teaching}}{OIST, Japan}{}{Delivered lectures on using \textsc{MATLAB} as a research computing environment with emphasis on acceleration methods for numerical problems.}
%\cventry{2011--2012}{Laboratory teaching assistant}{\textit{Teaching}}{UCC, Ireland}{}{Demonstrator of experimental laboratory work in physics for undergraduates.}
\cventry{2010--2011}{Software developer}{}{IBM, Ireland}{}{Developed software for IBM WebSphere Portal Server. Experience with Java, XML, XLST, Python, Shell. Further information available upon request}
\cventry{2009}{Product engineer (intern)}{}{Analog Devices BV, Ireland}{}{Analysis of semiconductor device yields and failures. Further information available upon request}

\section{Relevant publications}
\cvline{2016}{\textit{``Topological defect dynamics of vortex lattices in Bose--Einstein condensates'', }{L.~J.~O'Riordan, Th.~Busch.} \href{http://journals.aps.org/pra/abstract/10.1103/PhysRevA.94.053603}{Phys. Rev. A 94, 053603 (2016)}. DOI: {10.1103/PhysRevA.94.053603}}
\cvline{2016}{\textit{``Moir\'e superlattice structures in kicked Bose-Einstein condensates'', }{L.~J.~O'Riordan, A.~C.~White, Th.~Busch.} \href{https://journals.aps.org/pra/abstract/10.1103/PhysRevA.93.023609}{Phys. Rev. A 93, 023609 (2016)}. DOI: {10.1103/PhysRevA.93.023609}}
\cvline{2013}{\textit{``Coherent transport by adiabatic passage on atom chips'', }{T.~Morgan, L.~J.~O'Riordan, N.~Crowley, Th.~Busch,} \href{http://journals.aps.org/pra/abstract/10.1103/PhysRevA.88.053618}{Phys. Rev. A 88, 053618 (2013)}. DOI: {10.1103/PhysRevA.88.053618}}

\iffalse
\section{Community service and outreach}
\cventry{2017--2020}{Member}{Board of Councilors}{OIST, Japan}{}{Elected as first alumnus to Board of Councilors by OIST Board of Governors.}{}
\cventry{2015--2016}{Chair}{Student Council}{OIST, Japan}{}{Elected by student body. Facilitated regular meetings between the student body and faculty. Improved institutional policies and conditions for students}{}
\cventry{2014--2017}{HPC advanced users group representative}{Quantum Systems Unit}{OIST, Japan}{}{Research unit representative to the OIST HPC and scientific computing team. Trained group members on the use of HPC software and advanced programming techniques}{}
\fi

%\iffalse
%\section{Competitions \& awards}
%\cventry{2013}{2nd place}{Euraxess Links Japan, Science Slam 2013}{Tokyo Institute of Technology, Japan}{}{Presentation title: ``Quantum Typhoons''}{}
%\cventry{2010}{Student of the Year}{School of Science}{WIT, Ireland}{}{}{}
%\cventry{2010}{Best undergraduate research project}{Department of Computing, Maths and Physics}{WIT, Ireland}{}{}{}
%\cventry{2010}{Earnshaw Medal Nominee}{Institute of Physics}{Ireland}{}{}
%\cventry{2007}{3rd place}{Robocode: Inter-varsity AI programming competition}{Ireland}{}{}
%\fi



\iffalse
\section{Presentations}
\cventry{Mar, 2016}{WQS 2016} {Conference}{Dunedin}{New Zealand}{Poster presentation: ``\textit{Dynamics of large vortex lattice carrying Bose--Einstein condensates}"}
\cventry{June, 2015}{ICOLS 2015} {Conference}{Sentosa}{Singapore}{Poster presentation: ``\textit{Moiré super-lattice in a kicked rapidly rotating Bose–Einstein condensate}"}
\cventry{Sept, 2014}{Coherent Quantum Dynamics} {Workshop}{Okinawa Institute of Science and Technology Graduate University}{Japan}{Poster presentation: ``\textit{Investigating vortex dynamics in harmonically trapped Bose-Einstein condensates}"}
\cventry{July, 2014}{Non-linear Dynamics, Dynamical Transitions and Instabilities in Classical and Quantum Systems}{Workshop}{ICTP}{Italy}{Poster presentation: ``\textit{Investigating vortex dynamics in harmonically trapped Bose-Einstein condensates}"}
\cventry{April, 2014}{C3QS 2014}{Conference}{Okinawa Institute of Science and Technology Graduate University}{Japan}{Poster presentation: ``\textit{Coherent transport by adiabatic passage on atom chips}"}
\cventry{Sept, 2013}{Manipulation of degenerate quantum gases}{Workshop}{\'{E}cole Pr\'{e}doctorale de Physique des Houches}{France}{Poster presentation: ``\textit{Coherent transport by adiabatic passage on atom chips}"}
\cventry{May, 2013}{C3QS 2013}{Conference}{Okinawa Institute of Science and Technology Graduate University}{Japan}{Poster presentation:  ``\textit{Simulating 3D atomic dynamics using a graphics processor accelerated split-operator method}"}
\cventry{Sept, 2012}{QuAMP 2012}{Workshop}{Queens University Belfast}{Northern Ireland}{Attendee}
\cventry{Jul, 2012}{QUACS 2012}{Workshop}{University of Nottingham}{UK}{Poster presentation: ``\textit{Simulations of 3D atomic dynamics using a GPU accelerated split-operator method}"}
\cventry{Mar, 2012}{Institute of Physics Ireland Spring Meeting 2012}{Workshop}{Royal College of Surgeons}{Ireland}{Poster presentation: ``\textit{Simulations of 3D atomic dynamics using a GPU accelerated split-operator method}"}
\cventry{Mar, 2012}{DPG Spring Meeting}{University of Stuttgart}{Germany}{Conference}{Oral presentation: ``\textit{Simulations of 3D atomic dynamics using a GPU accelerated split-operator method}"}
\fi

\iffalse
\section{Natural languages}
\cvlanguage{English}{Native}{}
\cvlanguage{Irish}{Intermediate}{}
\cvlanguage{Japanese}{Basic}{}
\fi

\iffalse
\section{Competitions \& awards}
\cventry{2014}{Best Customer Interaction}{Kyued-Up Innovation Event}{Okinawa Institute of Science and Technology Graduate University, Japan}{Project title: ``Okinawa Science Discovery Center''}{\href{http://pullapproach.com}{http://pullapproach.com}}{}
\cventry{2013}{2nd place}{Euraxess Links Japan, Science Slam 2013}{Tokyo Institute of Technology}{Japan}{Talk title: ``Quantum Typhoons''}{}
\cventry{2010}{Student of the Year}{School of Science, Waterford Institute of Technology}{Ireland}{}{}
\cventry{2010}{Best undergraduate research project}{Department of Computing, Mathematics and Physics, Waterford Institute of Technology}{Ireland}{}{}
\fi

\iffalse
\section{References}
\cvline{PhD Supervisor}{Prof. Thomas Busch, Okinawa Institute of Science and Technology Graduate University, Okinawa, Japan. \newline{} tel: +81 (0)98 966 1588,  e-mail:\href{thomas.busch@oist.jp}{thomas.busch@oist.jp}}
\cvline{PhD academic mentor}{Prof. Nic Shannon, Okinawa Institute of Science and Technology Graduate University, Okinawa, Japan. \newline{}  {e-mail:\href{nic.shannon@oist.jp}{nic.shannon@oist.jp}}}
\cvline{Postdoctoral researcher}{Dr. Angela White, Okinawa Institute of Science and Technology Graduate University, Okinawa, Japan. \newline{}  {e-mail:\href{angela.white@oist.jp}{angela.white@oist.jp}}}
\fi

\vspace{1ex}
\begin{center}
\textbf{Additional information is available upon request.}
\end{center}

\clearpage

%\recipient{École Prédoctorale de Physique des Houches}{Centre de Physique des Houches, Côte des Chavants, F-74310 Les Houches, France.} % Letter recipient
%\date{\today} % Letter date
%\opening{Dear Sir or Madam,} % Opening greeting
%\closing{Sincerely,} % Closing phrase
%
%\makelettertitle % Print letter title
%
%{The topics covered on this year's programme are very relevant to my areas of study as part of my PhD thesis.
%During my first year of PhD studies at University College Cork (UCC) under Thomas Busch I have worked on coherent atomic control using atom-chip structures.
%Employing the matter-wave analogue of STIRAP, coherent tunneling via adiabatic passage (CTAP) as outlined by Eckert et al (doi:10.1016/j.optcom.2006.02.056), and extending the work carried out by
%O'Sullivan et al. (doi:10.1088/0031-8949/2010/T140/014029) on 2D trapping potentials, a fully experimentally realisable 3D system was devised.  Simulations of our proposed system showed verification of the CTAP process, wherein we were able to show transfer fidelities of 99.8\% for counter-intuitive waveguide arrangements, and verification of the difficulties with direct tunneling via intuitive arrangements.
%I presented this work at DPG 2012, with the final version of the resulting paper currently being finalised.
%Upon completion of this body of work, I have commenced work on analysing vortex dynamics in Bose-Einstein condensates. I wish to investigate the dynamics of a vortex lattice structure wherein, given sufficient rotation, the vortices will form an Abrikosov lattice pattern in the ground-state. Given careful choice of lattice spacings, periodic pulsing of an optical lattice pattern may yield dynamics akin to that of a delta-kicked harmonic oscillator, and allow for a means of visualising quantum turbulent or chaotic behaviour. As part of this work I have
%developed a program for performing integration of the time dependent Gross-Pitaevskii equation which outperforms equivalent programs in Fortran, C and Matlab for achieving results in short timescales.
%More recently, having transferred to the graduate programme at Okinawa Institute of Science \& Technology (OIST), I have gained more variety in my studies. As a formal requirement, one must take coursework both inside and outside the area of proposed specialisation. I have spent time working in a cryo-electron microscopy biology group,
%and taken courses on protein crystallization and in single particle tomography and 3D molecular reconstruction. Given my diverse background,
%having spent some time in industry before pursuing PhD study, and resulting work carried out as part of the OIST PhD programme, I may offer a unique
%perspective on many topics. Having a background in high-performance computing I believe is also beneficial, and I may offer insight to many others
%who wish to perform simulations of complex physical systems in reasonable timescales. Lastly, this school would also offer the ability for me to build my academic network with the possibility of developing future collaborative efforts with other such attendees. It is for these reasons that I wish to attend this summer school.
%}
%
%\makeletterclosing % Print letter signature

\end{document}
