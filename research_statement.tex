\documentclass[12pt,a4paper,unicode]{moderncv}
\moderncvtheme[grey]{mlxdmoderncv}                % oldstyle banking casual classic
\usepackage[utf8]{inputenc}
\usepackage[scale=0.82]{geometry}
\setlength{\hintscolumnwidth}{2.5cm}

%\hypersetup{colorlinks,breaklinks,urlcolor=linkcolour, linkcolor=linkcolour}

\firstname{Lee James}
\familyname{O'Riordan}
%\title{BSc(Hons), AMInstP}
\address{Quantum Systems Unit, Okinawa Institute of Science and Technology Graduate University}{1919-1 Tancha, Onna-son, Okinawa, Japan 904-0495}
\mobile{+81 (0)80 6498 4323}
\email{lee.oriordan@oist.jp}
\social[linkedin]{loriordan}
\social[github]{mlxd}

\makeatletter

\begin{document}
    \opening{Dear Prof. Bruun,}
    \recipient{Prof. Georg Bruun}{Aarhus University, Denmark}
    \date{\today} % Letter date
    \closing{Sincerely,}
    \makelettertitle
{
    In this letter I will present a summary and statement of my research to date, as well as my future plans. My PhD was undertaken at the Okinawa Institute of Science and Technology Graduate University, Japan, and also briefly at University College Cork, Ireland, under the supervision of Prof. Thomas Busch. My area of research is in theoretical cold atomic physics, specifically on the dynamics of Bose--Einstein condensates. My PhD work aims to understand the dynamics of perturbations on quantum states with large values of angular momentum. I have worked primarily on understanding the dynamics of superfluids and quantum vortices. For these studies I have gained expertise in computational techniques that I know will be beneficial for a variety of problems.

    My thesis work concentrates on applying two types of perturbations to the equilibrium state of the rapidly rotating condensate system: i) the modification of the phonon spectrum of the condensate through the use of a kicked optical potential, without modification of the angular momentum; ii) the direct control of the topological excitations (vortices), and hence the angular momentum, which is performed with direct phase engineering of the wavefunction. Both methods are realisable using currently available state-of-the-art experimental control techniques. Using these techniques I have studied the creation of moir\'e superlattice patterns and the robustness of the vortex lattice system~[\href{http://journals.aps.org/pra/abstract/10.1103/PhysRevA.93.023609}{Phys. Rev. A 93, 023609 (2016)}], and have investigated order-to-disorder transitions and the creation of topological lattice defects~[\href{https://arxiv.org/abs/1608.07756}{arXiv:1608.07756}, accepted for publication in Physical Review A].

    %The use of a direct phase engineering approach shows itself to be a good candidate for investigating order-to-disorder transitions in a vortex lattice system. In a recent work I demonstrated the creation of stable topological defects in the vortex lattice which persisted for long times through the creation of vacancies in the vortex lattice. I classify the lattice disorder using orientational correlation functions, and also by the presence of vortices with non sixfold nearest neighbours using Delaunay triangulations. While not yet investigated, I intend to examine whether a hexatic phase can be created in the vortex lattice system using this approach.

    Over the course of my PhD I have also developed very capable numerical solvers for both linear and nonlinear Schr\"odinger systems using graphics processing units (GPUs) as computational accelerators. Using a variants of these codes, I have demonstrated 3D numerical integration of the Schr\"odinger equation for an experimentally realistic systems to investigate spatial adiabatic passage (SAP)~[\href{http://journals.aps.org/pra/abstract/10.1103/PhysRevA.88.053618}{Phys.~Rev.~A~88, 053618 (2013)}]. More recently, I have developed an open-sourced code for simulating vortex lattice dynamics in rapidly rotating Bose--Einstein condensates in two-dimensions, titled ``GPUE''~[\href{https://github.com/mlxd/gpue}{GitHub:mlxd/gpue}], which was used in my works on vortex lattices. The resulting software methods have been independently performance tested, and have been shown to outperform well-established and mature software packages in C/C++ and \textsc{MATLAB}~[http://peterwittek.com/gpe-comparison.html].

    Recently, I have begun a collaboration with another researcher on the code to allow for solution of the three-dimensional Gross--Pitaevskii solver using graphics processors. The code is currently being extending with the ability to include gauge potentials, and will in time allow for the investigation of vortex knots, tangles, and reconnections in 3D condensates. The use of gauge potentials will also allow for the investigations into artificial magnetism, with the potential for developing simulators of crystalline condensed matter physics. Quantum turbulence using vortices in two and three dimensional condensates can also be directly accessed using gauge potentials to define well-ordered starting configurations. While I have not worked with superfluid Fermi systems, nor polaron BEC systems, I would like to gain some understanding of the behaviour of these such systems to allow for their simulation also.

    To further the suite I aim to implement a graphics processing unit (GPU) enabled numerical Bogoliubov-de Gennes solver. Extending the tools to investigate Fermi-superfluid properties will allow for the toolset to examine many new ideas and projects in quantum superfluid systems. If the resources are available, I also aim to port this suite of tools to the Julia programming language for ease of use, and leverage the Intel Xeon Phi platform which can remove the need for distributed software implementation. With this extended toolset, examining the dynamics of a variety of complex quantum systems in detail will be possible. Having spent some time in industry before pursuing PhD study, I have observed that simulations involving GPUs and other accelerators have become common place in research. I anticipate that it will be imperative to develop the skills and techniques necessary to utilise these advanced devices to remain competitive.

    %The applicability of machine learning, artificial intelligence, and optimal control techniques can also potentially allow for improvements to both the numerical simulations, as well as the understanding of the resulting dynamical behaviours. I intend to spend devote some time investigating the applicability of these methods to simulating quantum systems. Some potential works that can take advantage of these tools and techniques include investigations of quantum turbulence starting from well ordered systems, artificial magnetism, and defect and impurity dynamics in superfluid systems. Quantum turbulence using vortices in two and three dimensional condensates will be directly accessible using gauge potentials to define well-ordered starting configurations. The use of gauge potentials will also allow for the investigations into artificial magnetism, with the potential for developing simulators of crystalline condensed matter physics.

    While I have listed my work on quantum dynamics and high performance numerical methods, my interests for future work are not limited to these systems. I have also gained a variety of experience in many areas outside my specialisation, having spent time working with cryo-electron microscopy, protein crystallization, single particle tomography, 3D molecular reconstruction, fluid mechanics, and optics. I am curious to examine any potential overlaps within these areas and my own.

    The computational experience I can bring to this position can allow for a variety of numerically challenging problems to be solved, and can allow for a variety of complex dynamical scenarios to be investigated in a short times.
}

\makeletterclosing
\end{document}
