\documentclass[11pt,a4paper,unicode]{moderncv}

\moderncvtheme[grey]{classic}                % oldstyle banking casual classic

\usepackage[utf8]{inputenc}

\usepackage[scale=0.88]{geometry}
\setlength{\hintscolumnwidth}{3cm}

\usepackage{hyperref}								% to use hyperlinks
\definecolor{linkcolour}{rgb}{0,0.2,0.6}			% hyperlinks setup
\hypersetup{colorlinks,breaklinks,urlcolor=linkcolour, linkcolor=linkcolour}

\firstname{Lee James}
\familyname{O'Riordan}
%\title{BSc(Hons), AMInstP}               % optional, remove the line if not wanted
%\address{Quantum Systems Unit, Okinawa Institute of Science and Technology Graduate University}{1919-1 Tancha, Onna-son, Okinawa, Japan 904-0495}   % optional, remove the line if not wanted
\mobile{+81 (0)80 6498 4323}
%\phone{+81 (0)98 966 1588}
\email{lee.oriordan@oist.jp}
\extrainfo{\href{http://groups.oist.jp/qsu}{http://groups.oist.jp/qsu}}
\social[linkedin]{loriordan}
\social[github]{mlxd}

\makeatletter
\renewcommand*{\bibliographyitemlabel}{\@biblabel{\arabic{enumiv}}}
\makeatother

%\nopagenumbers{}

%----------------------------------------------------------------------------------
%            content
%----------------------------------------------------------------------------------
\begin{document}
\maketitle
\vspace*{-30pt}
\section{Education}
\cventry{2012--Present}{PhD in Science (Physics)}{Quantum Systems Unit, Okinawa Institute of Science and Technology Graduate University}{OIST, Japan}{Expected graduation: March, 2017}{}
\cventry{2011--2012}{PhD in Physics}{Ultracold Quantum Gases group, University College Cork}{UCC, Ireland}{}{Transferred to OIST, Japan after completing first year.}
\cventry{2006--2010}{BSc (Hons) in Physics with Computing}{Waterford Institute of Technology}{WIT, Ireland}{First class honours}{}

\section{Experience}
\cventry{2012--Present}{Research assistant (PhD)}{\textit{Education}}{OIST, Japan}{}{Researching theory of cold atomic quantum systems. Experience in linear algebra, differential equations, analysis, probability theory, and numerical algorithms. Computational modeling and numerical solution of problems using \textsc{MATLAB}, Mathematica, C/C++, CUDA, Python, and pencil \& paper.}{}
\cventry{2013--2015}{ \textsc{MATLAB} instructor}{\textit{Teaching}}{OIST, Japan}{}{Delivered lectures on using \textsc{MATLAB} as a research computing environment. Emphasised acceleration methods for numerical linear algebra problems.}
\cventry{2011--2012}{Laboratory teaching assistant}{\textit{Teaching}}{UCC, Ireland}{}{Demonstrator of experimental laboratory work in physics for undergraduates.}
\cventry{2010--2011}{Software engineer}{\textit{Working}}{IBM, Ireland}{}{Developed software for IBM WebSphere Portal Server. Further information available upon request.}
\cventry{2009}{Product engineer}{\textit{Internship}}{Analog Devices BV, Ireland}{}{Worked on failure and yield analysis of semiconductor devices. Further information available upon request.}

\section{Publications}
\cvline{2013}{\textit{``Coherent transport by adiabatic passage on atom chips'', }{T.~Morgan, L.~J.~O'Riordan, N.~Crowley, Th.~Busch,} \href{http://journals.aps.org/pra/abstract/10.1103/PhysRevA.88.053618}{Phys. Rev. A 88, 053618}. DOI: {10.1103/PhysRevA.88.053618}}
\cvline{2016}{\textit{``Moir\'e superlattice structures in kicked Bose-Einstein condensates'', }{L.~J.~O'Riordan, A.~C.~White, Th.~Busch.} \href{https://journals.aps.org/pra/abstract/10.1103/PhysRevA.93.023609}{Phys. Rev. A 93, 023609}. DOI: {10.1103/PhysRevA.93.023609}}
\cvline{2016}{\textit{``Topological defect dynamics of vortex lattices in Bose--Einstein condensates'', }{L.~J.~O'Riordan, Th.~Busch.} \href{https://arxiv.org/abs/1608.07756}{arXiV:1608.07756}. {Submitted for publication in Physical Review A.}}

%\iffalse
\section{University service}
\cventry{2015-2016}{Chair}{Student Council}{OIST, Japan}{}{Elected by student body. Facilitated regular meetings between the student body and faculty. Improved institutional policies and conditions for students}{}
\cventry{2014-}{HPC advanced users group representative}{Quantum Systems Unit}{OIST, Japan}{}{Research unit representative to the OIST HPC and scientific computing team. Trained group members on the use of HPC software and advanced programming techniques}{}
\cventry{2013-2016}{President}{Music Club}{OIST, Japan}{}{}{}
\cventry{2013-2016}{ITSSC member}{OIST}{}{}{Member of university task force for updating IT policies and compliance.}{}
\cventry{2013-2015}{Lecturer \& speaker}{Outreach}{OIST, Japan}{}{}{}
%\fi

%\iffalse
\section{Competitions \& awards}
\cventry{2013}{2nd place}{Euraxess Links Japan, Science Slam 2013}{Tokyo Institute of Technology, Japan}{}{Presentation title: ``Quantum Typhoons''}{}
\cventry{2010}{Student of the Year}{School of Science}{WIT, Ireland}{}{}{}
\cventry{2010}{Best undergraduate project}{Department of Computing, Maths and Physics}{WIT, Ireland}{}{}{}
%\cventry{2010}{Earnshaw Medal Nominee}{Institute of Physics}{Ireland}{}{}
%\cventry{2007}{3rd place}{Robocode: Inter-varsity AI programming competition}{Ireland}{}{}
%\fi

\iffalse
\section{Computer language proficiencies \& skills}
\cvline{Basic}{\small HTML, C\#, Visual Basic, MPI}
\cvline{Intermediate}{\small OpenCL, Python, Java, OpenMP, Mathematica, Maple, Bash/Zsh, {\LaTeX}, Julia.}
\cvline{Expert}{\small C/C++, CUDA, \textsc{MATLAB}.}
\cvline{}{}
\cvline{Tools}{\small GCC, GDB, CuFFT, Nvidia Visual Profiler, Nvidia Nsight, GNU Scientific Library, Raspberry Pi.}
\cvline{OS \& Applications}{\small Windows, Linux/Unix, OS X, Office (Excel, Word, Powerpoint), iWork (Numbers, Pages, Keynote), LibreOffice/OpenOffice, GIMP, EndNote, OneNote.}
\fi


\section{Software projects (GitHub)}
\cvline{GPUE}{Architect \& developer of GPUE: GPU-enabled Gross--Pitaevskii equation solver; a pseudospectral linear and nonlinear partial differential equation solver and simulation tool. Languages: CUDA, C/C++, Python, \textsc{MATLAB}, Shell. Available at \href{http://github.com/mlxd/gpue}{mlxd/gpue}.}
\cvline{GraphIt}{Graph generation library for 2D spatial data. Language: C++11. Available at \href{http://github.com/mlxd/GraphIt}{mlxd/GraphIt}. }
\cvline{hexacorr}{Correlation functions for 2D spatial data. Language: \textsc{MATLAB}. Available at \href{http://github.com/mlxd/hexacorr}{GH/mlxd/hexacorr}.}
\cvline{Sparsey.jl}{Finite difference sparse matrix numerical diagonalizer of the quantum harmonic oscillator Hamiltonian. Language: Julia. Available at \href{http://github.com/mlxd/sparsey.jl}{mlxd/sparsey.jl}.}


\iffalse
\section{Presentations}
\cventry{Mar, 2016}{WQS 2016} {Conference}{Dunedin}{New Zealand}{Poster presentation: ``\textit{Dynamics of large vortex lattice carrying Bose--Einstein condensates}"}
\cventry{June, 2015}{ICOLS 2015} {Conference}{Sentosa}{Singapore}{Poster presentation: ``\textit{Moiré super-lattice in a kicked rapidly rotating Bose–Einstein condensate}"}
\cventry{Sept, 2014}{Coherent Quantum Dynamics} {Workshop}{Okinawa Institute of Science and Technology Graduate University}{Japan}{Poster presentation: ``\textit{Investigating vortex dynamics in harmonically trapped Bose-Einstein condensates}"}
\cventry{July, 2014}{Non-linear Dynamics, Dynamical Transitions and Instabilities in Classical and Quantum Systems}{Workshop}{ICTP}{Italy}{Poster presentation: ``\textit{Investigating vortex dynamics in harmonically trapped Bose-Einstein condensates}"}
\cventry{April, 2014}{C3QS 2014}{Conference}{Okinawa Institute of Science and Technology Graduate University}{Japan}{Poster presentation: ``\textit{Coherent transport by adiabatic passage on atom chips}"}
\cventry{Sept, 2013}{Manipulation of degenerate quantum gases}{Workshop}{\'{E}cole Pr\'{e}doctorale de Physique des Houches}{France}{Poster presentation: ``\textit{Coherent transport by adiabatic passage on atom chips}"}
\cventry{May, 2013}{C3QS 2013}{Conference}{Okinawa Institute of Science and Technology Graduate University}{Japan}{Poster presentation:  ``\textit{Simulating 3D atomic dynamics using a graphics processor accelerated split-operator method}"}
\cventry{Sept, 2012}{QuAMP 2012}{Workshop}{Queens University Belfast}{Northern Ireland}{Attendee}
\cventry{Jul, 2012}{QUACS 2012}{Workshop}{University of Nottingham}{UK}{Poster presentation: ``\textit{Simulations of 3D atomic dynamics using a GPU accelerated split-operator method}"}
\cventry{Mar, 2012}{Institute of Physics Ireland Spring Meeting 2012}{Workshop}{Royal College of Surgeons}{Ireland}{Poster presentation: ``\textit{Simulations of 3D atomic dynamics using a GPU accelerated split-operator method}"}
\cventry{Mar, 2012}{DPG Spring Meeting}{University of Stuttgart}{Germany}{Conference}{Oral presentation: ``\textit{Simulations of 3D atomic dynamics using a GPU accelerated split-operator method}"}
\fi

\iffalse
\section{Natural languages}
\cvlanguage{English}{Native}{}
\cvlanguage{Irish}{Intermediate}{}
\cvlanguage{Japanese}{Basic}{}
\fi

\iffalse
\section{Competitions \& awards}
\cventry{2014}{Best Customer Interaction}{Kyued-Up Innovation Event}{Okinawa Institute of Science and Technology Graduate University, Japan}{Project title: ``Okinawa Science Discovery Center''}{\href{http://pullapproach.com}{http://pullapproach.com}}{}
\cventry{2013}{2nd place}{Euraxess Links Japan, Science Slam 2013}{Tokyo Institute of Technology}{Japan}{Talk title: ``Quantum Typhoons''}{}
\cventry{2010}{Student of the Year}{School of Science, Waterford Institute of Technology}{Ireland}{}{}
\cventry{2010}{Best undergraduate project}{Department of Computing, Maths and Physics, Waterford Institute of Technology}{Ireland}{}{}
\fi

%\section{Referees}
%\cvline{PhD Supervisor}{Prof. Thomas Busch, Okinawa Institute of Science and Technology, Okinawa, Japan. \newline{} tel: +81 (0)98 966 1588 \hspace*{10pt} e-mail:\href{thomas.busch@oist.jp}{thomas.busch@oist.jp}}
%\cvline{PhD academic mentor}{Prof. Nic Shannon, Okinawa Institute of Science and Technology, Okinawa, Japan. \newline{} \hspace*{10pt} e-mail:\href{nic.shannon@oist.jp}{nic.shannon@oist.jp}}

\vspace{1ex}
\begin{center}
\textbf{Please contact me for references.}
\end{center}

\clearpage

%\recipient{École Prédoctorale de Physique des Houches}{Centre de Physique des Houches, Côte des Chavants, F-74310 Les Houches, France.} % Letter recipient
%\date{\today} % Letter date
%\opening{Dear Sir or Madam,} % Opening greeting
%\closing{Sincerely,} % Closing phrase
%
%\makelettertitle % Print letter title
%
%{The topics covered on this year's programme are very relevant to my areas of study as part of my PhD thesis.
%During my first year of PhD studies at University College Cork (UCC) under Thomas Busch I have worked on coherent atomic control using atom-chip structures.
%Employing the matter-wave analogue of STIRAP, coherent tunneling via adiabatic passage (CTAP) as outlined by Eckert et al (doi:10.1016/j.optcom.2006.02.056), and extending the work carried out by
%O'Sullivan et al. (doi:10.1088/0031-8949/2010/T140/014029) on 2D trapping potentials, a fully experimentally realisable 3D system was devised.  Simulations of our proposed system showed verification of the CTAP process, wherein we were able to show transfer fidelities of 99.8\% for counter-intuitive waveguide arrangements, and verification of the difficulties with direct tunneling via intuitive arrangements.
%I presented this work at DPG 2012, with the final version of the resulting paper currently being finalised.
%Upon completion of this body of work, I have commenced work on analysing vortex dynamics in Bose-Einstein condensates. I wish to investigate the dynamics of a vortex lattice structure wherein, given sufficient rotation, the vortices will form an Abrikosov lattice pattern in the ground-state. Given careful choice of lattice spacings, periodic pulsing of an optical lattice pattern may yield dynamics akin to that of a delta-kicked harmonic oscillator, and allow for a means of visualising quantum turbulent or chaotic behaviour. As part of this work I have
%developed a program for performing integration of the time dependent Gross-Pitaevskii equation which outperforms equivalent programs in Fortran, C and Matlab for achieving results in short timescales.
%More recently, having transferred to the graduate programme at Okinawa Institute of Science \& Technology (OIST), I have gained more variety in my studies. As a formal requirement, one must take coursework both inside and outside the area of proposed specialisation. I have spent time working in a cryo-electron microscopy biology group,
%and taken courses on protein crystallization and in single particle tomography and 3D molecular reconstruction. Given my diverse background,
%having spent some time in industry before pursuing PhD study, and resulting work carried out as part of the OIST PhD programme, I may offer a unique
%perspective on many topics. Having a background in high-performance computing I believe is also beneficial, and I may offer insight to many others
%who wish to perform simulations of complex physical systems in reasonable timescales. Lastly, this school would also offer the ability for me to build my academic network with the possibility of developing future collaborative efforts with other such attendees. It is for these reasons that I wish to attend this summer school.
%}
%
%\makeletterclosing % Print letter signature

\end{document}
